\nonstopmode{}
\documentclass[a4paper]{book}
\usepackage[times,inconsolata,hyper]{Rd}
\usepackage{makeidx}
\usepackage[utf8]{inputenc} % @SET ENCODING@
% \usepackage{graphicx} % @USE GRAPHICX@
\makeindex{}
\begin{document}
\chapter*{}
\begin{center}
{\textbf{\huge Package `tempConvert'}}
\par\bigskip{\large \today}
\end{center}
\inputencoding{utf8}
\ifthenelse{\boolean{Rd@use@hyper}}{\hypersetup{pdftitle = {tempConvert: temp convert}}}{}
\begin{description}
\raggedright{}
\item[Type]\AsIs{Package}
\item[Title]\AsIs{temp convert}
\item[Version]\AsIs{0.1.0}
\item[Author]\AsIs{A.Ahmed}
\item[Maintainer]\AsIs{The package maintainer }\email{yourself@somewhere.net}\AsIs{}
\item[Description]\AsIs{Converts temps from °C to °F and vice versa.}
\item[License]\AsIs{MIT}
\item[Encoding]\AsIs{UTF-8}
\item[RoxygenNote]\AsIs{7.3.1}
\item[NeedsCompilation]\AsIs{no}
\end{description}
\Rdcontents{\R{} topics documented:}
\inputencoding{utf8}
\HeaderA{to\_c}{Convert Fahrenheit to Celsius}{to.Rul.c}
%
\begin{Description}
This function converts a temperature from Fahrenheit to Celsius.
\end{Description}
%
\begin{Usage}
\begin{verbatim}
to_c(f)
\end{verbatim}
\end{Usage}
%
\begin{Arguments}
\begin{ldescription}
\item[\code{f}] A numeric value representing the temperature in Fahrenheit.
\end{ldescription}
\end{Arguments}
%
\begin{Value}
The temperature in Celsius.
\end{Value}
\inputencoding{utf8}
\HeaderA{to\_f}{Convert Celsius to Fahrenheit}{to.Rul.f}
%
\begin{Description}
This function converts a temperature from Celsius to Fahrenheit.
\end{Description}
%
\begin{Usage}
\begin{verbatim}
to_f(c)
\end{verbatim}
\end{Usage}
%
\begin{Arguments}
\begin{ldescription}
\item[\code{c}] A numeric value representing the temperature in Celsius.
\end{ldescription}
\end{Arguments}
%
\begin{Value}
The temperature in Fahrenheit.
\end{Value}
\printindex{}
\end{document}
